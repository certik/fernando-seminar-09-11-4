%\documentclass{beamer}
\documentclass[handout]{beamer}

\usepackage[utf8]{inputenc}

\usetheme[bullet=circle,
          titleline=true,
          pageofpages=of,
          alternativetitlepage=true]{Torino}

\usepackage{color}

\usepackage{ragged2e}
\usepackage{hyphenat}

\usepackage{amsmath}
\usepackage{amsfonts}
\usepackage{amsthm}

%\usepackage{sympytex}
\usepackage{pythonhighlight}

\definecolor{MyGreen}{rgb}{0.40,0.80,0.20}

\title{SymPy --- a library for symbolic mathematics \linebreak in pure Python}
\author{Ondřej Čertík \texttt{<ondrej@certik.cz>}}
\institute[PWR]{University of Nevada, Reno\linebreak SymPy Development Team}
\date{August 19, 2009}

\newenvironment{jblock}[1]{
    \begin{block}{#1}\justifying\nohyphens
}{
    \end{block}
}

\begin{document}

\setbeamercovered{transparent}

\frame{\titlepage}

\begin{frame}[fragile]
    \frametitle{SymPy}

    \begin{itemize}
        \item A pure Python library for symbolic mathematics
        %\item A \structure<2>{pure} \structure<3>{Python} \structure<4>{library}
        %      for \structure<5>{symbolic} \structure<6>{mathematics}
    \end{itemize}

    \onslide<2->
    \begin{python}
  >>> from sympy import *
  >>> x = Symbol('x')

  >>> limit(sin(pi*x)/x, x, 0)
  pi

  >>> integrate(x + sinh(x), x)
  (1/2)*x**2 + cosh(x)

  >>> diff(_, x)
  x + sinh(x)
    \end{python}
\end{frame}

\begin{frame}[fragile]
    \frametitle{Capabilities}
    \framesubtitle{What SymPy can do}

    \begin{columns}
        \begin{column}[l]{0.5\textwidth}
            \begin{itemize}
                \item core functionality
                    \begin{itemize}
                        \item differentiation, truncated series
                        \item pattern matching, substitutions
                        \item non--commutative algebras
                        \item assumptions engine, logic
                    \end{itemize}
                \item symbolic \ldots
                    \begin{itemize}
                        \item integration, summation
                        \item limits
                    \end{itemize}
                \item polynomial algebra
                    \begin{itemize}
                        \item Gröbner bases computation
                        \item multivariate factorization
                    \end{itemize}
                \item matrix algebra
            \end{itemize}
        \end{column}
        \begin{column}[r]{0.5\textwidth}
            \begin{itemize}
                \item equations solvers
                    \begin{itemize}
                        \item algebraic, transcendental
                        \item recurrence, differential
                    \end{itemize}
                \item systems solvers
                    \begin{itemize}
                        \item linear, polynomial
                    \end{itemize}
                \item pretty--printing
                    \begin{itemize}
                        \item Unicode, ASCII
                        \item LaTeX, MathML
                    \end{itemize}
                \item 2D \& 3D plotting
                \item \structure{\ldots}
            \end{itemize}
        \end{column}
    \end{columns}
\end{frame}

\begin{frame}[fragile]
    \frametitle{Miscellaneous}
    %\framesubtitle{What SymPy can do}

    \begin{columns}
        \begin{column}[l]{0.5\textwidth}
            \begin{itemize}
                \item advantages of pure python
                    \begin{itemize}
                        \item jython (sympy can be used in java applications)
                        \item useful for testing python implementations: 1565 tests in 139 files on 21584 lines (pypy, jython, unladen swallow)
                        \item google app engine
                        \item iphone
                        \item easy to deploy on windows
                    \end{itemize}
                \item easy to use in web applications
                    \begin{itemize}
                        \item google app engine
                        \item http://live.sympy.org/
                        \item http://gamma.sympy.org/
                    \end{itemize}
            \end{itemize}
        \end{column}
        \begin{column}[r]{0.5\textwidth}
            \begin{itemize}
                \item finite element solvers
                    \begin{itemize}
                        \item defining equations for finite element (and other) solvers
                        \item generating (C/C++) shape functions (Legendre, Lobatto, ...)
                        \item \structure{\ldots}
                    \end{itemize}
                \item other usages
                    \begin{itemize}
                        \item calculate things symbolically, in conjunction
with numerics (numpy, scipy)
                        \item physics (quantum mechanics, general relativity, ...)
                        \item teaching (calculus, numerics, ...)
                        \item \structure{\ldots}
                    \end{itemize}
            \end{itemize}
        \end{column}
    \end{columns}
\end{frame}

\end{document}

